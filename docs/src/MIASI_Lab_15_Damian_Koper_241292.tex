\documentclass[12pt, a4paper]{article}
\usepackage{scrextend}
\usepackage[utf8]{inputenc}
\usepackage[polish]{babel}
\usepackage[T1]{fontenc}%polskie znaki
\usepackage[utf8]{inputenc}%polskie znaki
\usepackage{geometry}
\usepackage{float}
\usepackage{enumitem}
\usepackage{hyperref}
\usepackage{graphicx}
\usepackage{amsmath}
\usepackage{tabularx}
\usepackage{pdflscape}
\usepackage{listings}

\lstset{
  literate={ą}{{\k a}}1
  		     {Ą}{{\k A}}1
           {ż}{{\. z}}1
           {Ż}{{\. Z}}1
           {ź}{{\' z}}1
           {Ź}{{\' Z}}1
           {ć}{{\' c}}1
           {Ć}{{\' C}}1
           {ę}{{\k e}}1
           {Ę}{{\k E}}1
           {ó}{{\' o}}1
           {Ó}{{\' O}}1
           {ń}{{\' n}}1
           {Ń}{{\' N}}1
           {ś}{{\' s}}1
           {Ś}{{\' S}}1
           {ł}{{\l}}1
           {Ł}{{\L}}1
}
\lstset{
language=C,
basicstyle=\small\sffamily,
numbers=left,
numberstyle=\tiny,
frame=tb,
columns=fullflexible,
showstringspaces=false,
breaklines=true
}
\lstset{emptylines=0}
\renewcommand{\baselinestretch}{1.5}


\begin{document}

\begin{flushleft}
    Damian Koper \textbf{241292} \\
\end{flushleft}
\vspace{1cm}
{
    \centering
    {\Huge\scshape\bfseries Modelowanie i analiza systemów informatycznych }\\
    \large{Logika Temporalna i Automaty Czasowe - konstrukcja i weryfikacja automatów NuSMV do analizy programu.}\\
    \vspace{0.5cm}
}
\newcounter{ex}
\setcounter{ex}{0}

\newcounter{fm}
\setcounter{fm}{0}

\newcommand{\ex}[1]{
    \refstepcounter{ex}{
        \noindent\normalfont\Large\bfseries Zadanie \arabic{ex}.
    } \\
    #1
}

\newcommand{\fm}[4]{
    \refstepcounter{fm}{
        \noindent\normalfont\large\bfseries Formuła \arabic{fm} \normalsize\normalfont
        \begin{itemize}
            \item[UPPALL:] \texttt{#1}
            \item[CTL:] #2
            \item[Opis:] #3
            \item[Wynik:] #4
        \end{itemize}
    }
    \vspace{1cm}
}

\ex{Liczba liczb pierwszych}
\lstinputlisting[language=C]{../../lab15/ex_1_2.smv}

\end{document}
