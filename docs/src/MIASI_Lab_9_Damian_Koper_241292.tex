\documentclass[12pt]{article}
\usepackage{scrextend}
\usepackage[utf8]{inputenc}
\usepackage[polish]{babel}
\usepackage[T1]{fontenc}%polskie znaki
\usepackage[utf8]{inputenc}%polskie znaki
\usepackage{geometry}
\usepackage{float}
\usepackage{enumitem}
\usepackage{hyperref}
\usepackage{graphicx}
\usepackage{amsmath}

\renewcommand{\baselinestretch}{1.5}


\begin{document}

\begin{flushleft}
    Damian Koper \textbf{241292} \\
\end{flushleft}
\vspace{1cm}
{
    \centering
    {\Huge\scshape\bfseries Modelowanie i analiza systemów informatycznych }\\
    \large{Sieci Petriego - konstrukcja sieci Petriego z łukami czasowymi}\\
    \vspace{0.5cm}
}
\newcounter{ex}
\setcounter{ex}{0}
\newcommand{\ex}[1]{
    \refstepcounter{ex}{
        \noindent\normalfont\Large\bfseries Zadanie \arabic{ex}.
    } \\
    #1
}
\clearpage
\ex{Przekształcenie sieci - tramwaje}
\vspace{-0.5cm}

\begin{figure}[H]
    \centering
    \includegraphics[width=0.7\linewidth]{../../lab9/ex_1-timed}
    \caption{Semafory naprzemienne. Nie można przełączyć automatycznie z $p_{11}$ na $p_{10}$ semafora prawego, kiedy w $p_{2}$ pojawi się tramwaj.}
\end{figure}

\clearpage

\ex{Rozudowa sieci}
\begin{figure}[H]
    \centering
    \includegraphics[width=\linewidth]{../../lab9/ex_2}
    \caption{Sieć symulująca sygnalizację świetlną.}
\end{figure}

\end{document}